%%%%%%%%%%%%%%%%%%%%%%%%%%%%%%%%%%%%

\section{Poisson distribution}

%%%%%%%%%%%%%%%%%%%%%%%%%%%%%%%%%%%%

\begin{frame}
\frametitle{Poisson distribution}

\begin{itemize}

\item The \hl{Poisson distribution} is often useful for estimating the number of rare events in a large population over a short unit of time for a fixed population if the individuals within the population are independent.

\item The \hl{rate} for a Poisson distribution is the average number of occurrences in a mostly-fixed population per unit of time, and is typically denoted by \mathhl{\lambda}.

\item Using the rate, we can describe the probability of observing exactly $k$ rare events in a single unit of time.

\end{itemize}

\vfill

\formula{Poisson distribution}
{
P(observe $k$ rare events) = $\frac{\lambda^k e^{-\lambda}}{k!}$, \\
where $k$ may take a value 0, 1, 2, and so on, and $k!$ represents $k$-factorial. The letter $e \approx 2.718$ is the base of the natural logarithm. \\

The mean and standard deviation of this distribution are $\lambda$ and $\sqrt{\lambda}$, respectively.
}

\end{frame}

%%%%%%%%%%%%%%%%%%%%%%%%%%%%%%%%%%%%

\begin{frame}

\dq{Suppose that in a rural region of a developing country electricity power failures occur following a Poisson distribution with an average of 2 failures every week. Calculate the probability that in a given week the electricity fails only once.}

\pause

Given $\lambda = 2$.

\pause

\begin{eqnarray*}
P(\text{only 1 failure in a week}) &=& \frac{2^1 \times e^{-2}}{1!} \\
\pause
&=& \frac{2 \times e^{-2}}{1} \\
\pause
&=& 0.27
\end{eqnarray*}

\end{frame}

%%%%%%%%%%%%%%%%%%%%%%%%%%%%%%%%%%%%

\begin{frame}

\dq{Suppose that in a rural region of a developing country electricity power failures occur following a Poisson distribution with an average of 2 failures every week. Calculate the probability that on a given \underline{day} the electricity fails three times.}

\pause

We are given the weekly failure rate, but to answer this question we need to first calculate the average rate of failure on a given day: $\lambda_{day} = \frac{2}{7} = 0.2857$. Note that we are assuming that the probability of power failure is the same on any day of the week, i.e. we assume independence.

\pause

\begin{eqnarray*}
P(\text{3 failures on a given day}) &=& \frac{0.2857^1 \times e^{-0.2857}}{3!} \\
\pause
&=& \frac{0.2857 \times e^{-0.2857}}{6} \\
\pause
&=& 0.0358
\end{eqnarray*}

\end{frame}

%%%%%%%%%%%%%%%%%%%%%%%%%%%%%%%%%%%%

\begin{frame}
\frametitle{Is it Poisson?}

\begin{itemize}

\item A random variable may follow a Poisson distribution if the event being considered is rare, the population is large, and the events occur independently of each other

\item However we can think of situations where the events are not really independent. For example, if we are interested in the probability of a certain number of weddings over one summer, we should take into consideration that weekends are more popular for weddings.

\item In this case, a Poisson model may sometimes still be reasonable if we allow it to have a different rate for different times; we could model the rate as higher on weekends than on weekdays.

\item The idea of modeling rates for a Poisson distribution against a second variable (day of the week) forms the
foundation of some more advanced methods called \hl{generalized linear models}. There are beyond the scope of this course, but we will discuss a foundation of linear models in Chapters 7 and 8.

\end{itemize}

\end{frame}

%%%%%%%%%%%%%%%%%%%%%%%%%%%%%%%%%%%%

\begin{frame}
\frametitle{Practice}

\pq{A random variable that follows which of the following distributions can take on values other than positive integers?}

\begin{enumerate}[(a)]
\item Poisson
\item Negative binomial
\item Binomial
\solnMult{Normal}
\item Geometric
\end{enumerate}

\end{frame}